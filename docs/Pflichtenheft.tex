\documentclass[a4paper]{article}

% Define page margine
\usepackage[left=2.5cm,right=2.5cm,top=2.5cm,bottom=2.5cm]{geometry}
\usepackage{fullpage}
\usepackage{amsmath,amssymb,amsthm,enumitem}

% Enable use of figures
\usepackage[pdftex]{graphicx}

% Set line spacing
\usepackage{setspace}
\linespread{1.2}

% Allow for hyphenation
\usepackage{hyphenat}
\hyphenation{over-view}

% Include url package
\usepackage{url}

\title{Electronic German Voting System}

\author{Authors: Maisa Ben Salah, Yecine Megdiche}

\begin{document}

\maketitle
\section{Objectives}
\subsection{Must-Meet Criteria}
The system must:
\begin{itemize}
\item manage votes electronically. This includes:
        \begin{itemize}
      \item Providing an interface to receive single or batched votes.
      \item Storing a single vote submitted by a voter or batched votes submitted by election officials.
      \end{itemize}
  \item store all the data relevant to the 2017 and 2021 elections. This includes:
    \begin{itemize}
      \item All cast votes
      \item All parties and candidates participating in the elections
      \item The result aggregates for the previous election on the region and state level
    \end{itemize}
  \item calculate and display election results for a given year. The seat distribution in the parliament is calculated according to the Sainte-Lag\"ue method and takes into consideration:
        \begin{itemize}
          \item Electoral Threshold (Sperrklausel)
          \item Leveling and Overhang Seats (Ausgleichs- und \"Uberhangsmandate)
          \item Minority Parties (Minderheitsparteien)
        \end{itemize}
  \item produce and display statistics regarding voting trends on the region, state, and country level.
  \item display comparative statistics between two elections regarding the seat distribution and voting trends between years on the region, state and country level.
  \item show party and candidacy information on the region, state, and country level.
  \item grant restricted permissions to election officials to add and modify party and candidacy information.
  \item allow exclusively eligible people to vote only once per election.
  \item protect the privacy of the data pertaining to the voters and their votes.

\end{itemize}

\subsection{Should-Meet Criteria}
The system should:
\begin{itemize}
    \item scale to fit different screen sizes of mobile devices.
    \item provide interactive election results.
\end{itemize}

\subsection{Can-Have Criteria}
The system can:
\begin{itemize}
  \item include in the detailed view of the candidates their photos.
  \item display a dotted graphical visualization of the election results showing the exact numbers of the seats distribution for each party in the parliament.
  \item provide an interactive map to filter statistics and information.
\end{itemize}

\section{Technical Details}

\begin{itemize}
  \item \texttt{PostgreSQL} will be used as a \texttt{DBMS}. On top of that,
        \texttt{sqitch} will be used to manage schema migrations.
  \item \texttt{PostgREST} will be used to generate a \textit{REST API}, that
        relies on \textit{JSON} as a request-response format.
  \item If needed, \texttt{NginX} might be used as a reverse proxy or a web
        server.
  \item The fronted will be created in \texttt{TypeScript} using the
        \texttt{React} framework. \texttt{Bootstrap} will be used as a
        \textit{CSS} framework.
  \item The whole system will be containerized using \texttt{Docker}.

\end{itemize}

\section{Glossary}

\begin{itemize}
  \item \textbf{Direct seats:} seats that are won through first votes.
  \item \textbf{List seats:} seats that are won through second votes.
  \item \textbf{Overhang seats:} seats that result from a party winning more direct seats than the number of seats that it is entitled to from its share of the second votes.
  \item \textbf{Leveling seats:} seats that are given to compensate for overhang seats.
  \item \textbf{Electoral Threshold:} the condition for a party to receive leveling, consisting in a party winning 5\% of the total second votes or getting 3 direct seats.
\end{itemize}

\end{document}
